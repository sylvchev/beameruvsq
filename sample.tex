\documentclass[10pt,hidelinks]{beamer}

\usepackage{uvsq}
\usepackage[T1]{fontenc}
\usepackage[latin1]{inputenc}
\usepackage{xcolor}

% To include code
\usepackage{listings} 
\lstset{
  language=python,
  basicstyle=\scriptsize, %\footnotesize
  keywordstyle=\color{black}\bfseries,
  commentstyle=\color{gray},
  stringstyle=\ttfamily,
  numbers=left, 
  numberstyle=\tiny,
  frame=single,
  showspaces=false,
  showstringspaces=false
}

% To change the hyperref package options
\hypersetup{
    pdftitle={Example for UVSQ beamer theme},    % title
    pdfauthor={Sylvain Chevallier},     % author
    pdfnewwindow=true,      % links in new window
    pdfkeywords={beamer, UVSQ, latex}, % list of keywords
}

% Macro to avoid numbering the appendix frames
\newcommand{\backupbegin}{
   \newcounter{framenumberappendix}
   \setcounter{framenumberappendix}{\value{framenumber}}
}
\newcommand{\backupend}{
   \addtocounter{framenumberappendix}{-\value{framenumber}}
   \addtocounter{framenumber}{\value{framenumberappendix}} 
}

% Macro to have only link to external website in color
\definecolor{bleublason}{RGB}{0,146,187}
\newcommand{\colorhref}[2]{\href{#1}{\color{bleublason}{#2}}}

\title[Short Title]{This is a title}
\subtitle{and its subtitle}

\author[H. Cover]{
  Harry Cover\\
  \texttt{hc@uvsq.fr}
}

\institute[UVSQ]{
  Universit\'e de Versailles Saint-Quentin
  }

\date[September 2016]{September 22, 2016}

\begin{document}

\begin{frame}[plain]
  \titlepage
\end{frame}

% To recall the outline before each new section, uncomment below
% \AtBeginSection[]
% {
%   \begin{frame}<beamer>
%     \frametitle{Outline}
%     \tableofcontents
%   \end{frame}
% }

\begin{frame}
  \frametitle{Outline}
  \tableofcontents
\end{frame}
\section{This is the first section}

\subsection{A first subsection}

\begin{frame}
  \frametitle{This is the frametitle}
  Some basic text

  \vfill

  \begin{itemize}
  \item This is a text item
  \item And another one
  \end{itemize}

  \vfill

  \begin{enumerate}
  \item A first item
  \item A second one
  \end{enumerate}

  \vfill

  \begin{description}
  \item[A label] A description with label
  \item[Another one] A second description
  \end{description}
\end{frame}

\subsection{A second subsection}

\begin{frame}
  \frametitle{Using blocks}

  \begin{block}{This is a block title}
    \begin{itemize}
    \item A first item
    \item A second one
    \end{itemize}
  \end{block}
  \vfill
  \begin{alertblock}{Another block title}
    An alert block
  \end{alertblock}
\end{frame}

\section{Some useful environment}

\subsection{Using columns for structure}

\begin{frame}
  \frametitle{A frame with columns}

  Some text before the columns environment

  \vfill

  \begin{columns}
    \begin{column}{0.5\linewidth}
      \begin{itemize}
      \item Something
      \item Something else
      \end{itemize}
    \end{column}
    \begin{column}{0.5\linewidth}
      Some text or image

      \begin{center}
        \pgfimage[interpolate=true,width=0.95\linewidth]{uvsqlogo-upsay-long}
      \end{center}
    \end{column}
  \end{columns}

  \vfill

    Some text after
\end{frame}

\subsection{Putting code in presentation}

\begin{frame}[fragile]
  \frametitle{Including code}

  \begin{itemize}
  \item   The listings package allows to include code in \LaTeX{} docs
  \item More information on \colorhref{https://www.ctan.org/pkg/listings}{https://www.ctan.org/pkg/listings}
  \end{itemize}

  \vfill

\begin{lstlisting}
def fib(n):
    """Print Fibonacci sequence up to n."""
    a, b = 0, 1
    while b < n:
        print (b, end=' ')
        a, b = b, a+b
\end{lstlisting}

\end{frame}

\subsection{Adding reference to appendix}

\begin{frame}
  \frametitle{Button to reference appendix}

  On the webpage about the  \colorhref{http://www.uvsq.fr/le-logo-de-l-uvsq-233680.kjsp?RH=ACCUEIL-FR}{UVSQ logo}
  \begin{itemize}
  \item Color choice are explained
  \item The font of the logo are indicated
  \item The choice of the emblem 
  \end{itemize}
  \vfill

  \pgfimage[interpolate=true,width=\linewidth]{uvsqlogo-upsay-long}

  \begin{flushright}
    \hyperlink{compinf}{\beamergotobutton{More information on graphic charter}}
    \hypertarget{compinfhome}{}
  \end{flushright}

\end{frame}

\section{And there is a last one}

\begin{frame}
  \frametitle{A slide with references}
  Link to the git repository:\\
  \colorhref{https://github.com/sylvchev/beameruvsq}{https://github.com/sylvchev/beameruvsq}
  \vfill
  You could also include bibliography in a slide like this:
  \begin{thebibliography}{GPL,knuth}
  \bibitem{GPL}
    Free Software Foundation
    \newblock {\em GNU General Public License} 
    \newblock \colorhref{http://fsf.org}{fsf.org}, 2007.
  \bibitem{knuth}
    Knuth, Donald E.
    \newblock {\em Computer Programming as an Art}
    \newblock {CACM, 1974}
  \end{thebibliography}
\end{frame}


\backupbegin %to avoid numbering the appendix frames
\begin{frame}[plain]
  \begin{center}
    \vfill
    {\Large Appendix}
    \vfill
  \end{center}
\end{frame}

\begin{frame}
  \frametitle{Some complementary information}
  
  All the information regarding the UVSQ graphic charter are \colorhref{http://www.uvsq.fr/jsp/saisie/liste_fichiergw.jsp?OBJET=DOCUMENT&CODE=1429111529765&LANGUE=0}{available here}

  \begin{itemize}
  \item Thorough explanations about the logo
  \item All colors of the theme
  \item RGB, CMJN and Pantone codes
  \end{itemize}

  \vfill

  \begin{flushright}
    \hyperlink{compinfhome}{\beamergotobutton{back}}
    \hypertarget{compinf}{}
  \end{flushright}

\end{frame}
\backupend

\end{document}
